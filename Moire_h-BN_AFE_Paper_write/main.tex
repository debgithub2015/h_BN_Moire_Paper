\documentclass[apsrev,prl,twocolumn,superscriptaddress]{revtex4-2}
\usepackage{amssymb,amsfonts,amsmath}
\usepackage{adjustbox}
\usepackage{hyperref}
\usepackage{color,ulem}
%\usepackage{xr}
\makeatletter
%\newcommand*{\addFileDependency}[1]{% argument=file name and extension
%  \typeout{(#1)}
%  \@addtofilelist{#1}
%  \IfFileExists{#1}{}{\typeout{No file #1.}}
%}
%\makeatother

%\newcommand*{\myexternaldocument}[1]{%
%    \externaldocument{#1}%
%    \addFileDependency{#1.tex}%
%    \addFileDependency{#1.aux}%
%}
%\myexternaldocument{SupplementalMaterial_8_24_2019}


%\usepackage[square]{natbib}
%\graphicspath{{figs/}}





\documentclass{article}
\usepackage[utf8]{inputenc}

\begin{document}


\title{Study of Anti-Ferroelectric interactions in Moir\'{e} $h-$BN : Bilayer to Multilayer}
\author{Alejandro Pacheco}
\affiliation{Department of Mechanical Engineering, }
\author{Debajit Chakraborty}
\affiliation{Department of Physics, University of Arkansas, Fayetteville, Arkansas 72701, United States}
\author{Pradeep Kumar}
\affiliation{Department of Physics, University of Arkansas, Fayetteville, Arkansas 72701, United States}
\author{Salvador Barraza-Lopez}
\email{sbarraza@uark.edu}
\affiliation{Department of Physics, University of Arkansas, Fayetteville, Arkansas 72701, United States}
\affiliation{Institute for Nanoscience and Engineering, University of Arkansas, Fayetteville, Arkansas 72701, USA}


\begin{abstract}
{Artificial stacking of two-dimensional (2D) hexagonal boron-nitride (h-BN) is a promising ferroelectric candidate to generate local  electrical polarization switch for storage and memory applications. Owing to atomistically uniform thickness, integrable to other van der Waal (vdW) material through weak vdW interaction, high insulating property, and with a large spontaneous polarization, multi-layer 2D h-BN showing promising interest towards ferroelectric applications. Here we thoroughly studied the effect of temperature and stain on the polarization interaction for the artificially stacked twisted patterns, popularly known as Moir\'{e} pattern, for bi- and multi-layers of h-BN of various sizes through classical molecular dynamics(MD) simulations. We report a critical temperature at which the polarization pinning effect between layers due to electrostatic interaction disappears. }
\end{abstract}

\maketitle

%%%%%%%%%%%%%%%%%%%%%%%%%%%%%%%%%%%%%%%%%%%%%%%%%%%
\section{Introduction}
With the light-speed technological advancement in this modern era, the rising demand of new technology to build local/atomistic electrical switch in the field of microelectronics[Ref], field-effect transistors (FET)[Ref], and memory and logic functionalities [Ref] made Ferroelectrics essential candidates. Ferroelectric materials are non-centrosymmetric, and carry a spontaneous polarization due to an effective out-of-plane dipole moment. However, this spontaneous electric polarization can be reversed by inverting the external electric field. Thus, the generated hysteric loop of electric displacement as a function of the electric field strength between opposite polarities provides a electric bi-stability, which make Ferroelectrics ideal candidates for the local polarization switches. Generally, the ferroelectricity is a temperature dependent phase and stable below a certain temperature (namely Curie Temperature, $T_c$[Ref]). Moreover, sliding between two layers can also result into charge exchange at the contact surface. This is known as triboelectric effect[Ref]. Conventionally Perovskite[Ref]/Ilmenite[Ref] metal-oxides, polymers having long carbon chains[Ref], charge-transfer complexes[Ref],hydrogen-bonded[Ref] compounds are natural ferroelectric candidates.

However, with the present-day trend of reduced dimensions of electronics at nano-scale level anticipates certain challenges[Ref]. With conventional ferroelectrics (perovskites) on thin-film structures suffers from  

Hist



Introduction to Ferroelectricity and its application

Moir\'{e} structure and advantage of h-BN 

Potential interaction, strain

Tribology and rotation


In the subsequent sections, we have discussed our findings. In section II, we elaborately discuss the generation of commensurated Moir\'{e} structure and subsequent temperature dependent simulation details. In section III we will discuss the effect of the electric polarization to the strain and rotation of anti-ferroelectric layers at various temperature. We conclude our  findings at section IV by showing that there exist a critical temperature at which the electrostatic binding disappears.

\end{document}
